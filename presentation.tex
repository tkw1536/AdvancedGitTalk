\documentclass{beamer}

%Packages
\usepackage[utf8]{inputenc}
\usepackage{hyperref}

% Code
\usepackage{listings}

\lstdefinestyle{ShellCmd}{
    language=bash,
    basicstyle=\fontsize{6.5pt}{7pt}\ttfamily,
    backgroundcolor=\color{gray!10},
    linewidth=0.9\linewidth,
    breaklines=true,
    mathescape=false,
    literate={\$}{{\textcolor{blue}{\$}}}1
}

\newcommand\shellcmd[1]{\colorbox{gray!10}{\lstinline[style=ShellCmd,mathescape]`#1`}}

% Hide navigation symbols, show git
\setbeamertemplate{navigation symbols}{}
\setbeamertemplate{footline}[frame number]

% Logos
\newcommand{\logoimage}[2]{\begingroup
\setbox0=\hbox{\includegraphics[height=#2]{#1}}%
\parbox{\wd0}{\box0}\endgroup\ }

% Have a fancy git logo
\newcommand{\git}{\logoimage{imgs/git_logo.png}{8pt}}

%META-INFORMATION
\title{Advanced \logoimage{imgs/git_logo}{40px}\\How it works (and why it is so complicated)}
\author{Tom Wiesing}
\institute{KWARC second hour talk}
\date{March 29, 2016}

\begin{document}
    %TITLEPAGE
    \frame{\titlepage}
    
    \begin{frame}{Overview}
      \begin{itemize}
        \item The basics (what you should already know)
        \begin{itemize}
          \item What is git?
          \item Working directory, index \& repository
        \end{itemize}
          
          \item Object Storage (the ``filesystem'' of git)
          \begin{itemize}
            \item BLOBs, Trees, Commits
          \end{itemize}
          
          \item References in git
          \begin{itemize}
            \item HEADs, Branches, TAGs
          \end{itemize}
                   
          \item Merging \& Remotes (the fun part)
          \begin{itemize}
            \item Merging \& Rebasing
            \item Remotes: Fetch, Push \& Pull
          \end{itemize}
          
          \item Conclusion (What git is and what it is not)
      \end{itemize}
    \end{frame}
    
    \begin{frame}{The basics (1): What is git?}
  \begin{columns}[onlytextwidth]
    \begin{column}{0.6\textwidth}
      \begin{itemize}
        \item git -- ``the stupid content tracker''
        \begin{itemize}
          \item open-source version control system
          \item fast, scalable, distributable
        \end{itemize}
        \item originally developed in 2005 for maintaining the linux kernel source code
      \end{itemize}
    \end{column}
    \begin{column}[t]{0.4\textwidth}
      \includegraphics[width=0.95\textwidth]{imgs/git_logo}
    \end{column}
  \end{columns}
\end{frame}

\begin{frame}{The basics (2): Working directory, index \& repository}
  \begin{columns}[onlytextwidth]
    \begin{column}{0.6\textwidth}
      \begin{itemize}
        \item git maintains multiple versions of a project
        \item each repository has
        \begin{itemize}
          \item a working directory (where files are editable)
          \item a staging area (also called index)
          \item a git directory (contains all the history of the repository)
        \end{itemize}
        \item basic commands
        \begin{itemize}
          \item git add, git commit, git checkout
        \end{itemize}
      \end{itemize}
    \end{column}
    \begin{column}[c]{0.4\textwidth}
      \includegraphics[width=0.95\textwidth]{imgs/git_local}
    \end{column}
  \end{columns}
\end{frame}
    \begin{frame}{Object Storage (1): Object overview}
  \begin{itemize}
    \item git is a key-value store
    \begin{itemize}
      \item keys = SHA-1 hashes
    \end{itemize}
    \item Three main types of objects:
    \begin{itemize}
      \item BLOBs (for file content)
      \item Trees (for storing a directory of files)
      \item Commits (to store multiple versions)
    \end{itemize}
  \end{itemize}
\end{frame}

\begin{frame}[fragile]{Object Storage (2): BLOBs}
  \begin{itemize}
    \item stores the \textit{content} of a file
    \item problem: no meta-information such as filename, path
  \end{itemize}
\begin{lstlisting}[style=ShellCmd]
$ echo 'test content' | git hash-object -w --stdin
d670460b4b4aece5915caf5c68d12f560a9fe3e4
\end{lstlisting}
\begin{lstlisting}[style=ShellCmd]
$ git cat-file -p d670460b4b4aece5915caf5c68d12f560a9fe3e4
test content
\end{lstlisting}
\begin{itemize}
  \item storing multiple versions of the same file is no problem
\end{itemize}

\begin{lstlisting}[style=ShellCmd]
$ echo 'version 1' > test.txt
$ git hash-object -w test.txt
83baae61804e65cc73a7201a7252750c76066a30
\end{lstlisting}

\begin{lstlisting}[style=ShellCmd]
$ echo 'version 2' > test.txt
$ git hash-object -w test.txt
1f7a7a472abf3dd9643fd615f6da379c4acb3e3a
\end{lstlisting}

\end{frame}

\begin{frame}[fragile]{Object Storage (3): BLOBs continued}
  \begin{itemize}
    \item we can checkout each version individually
  \end{itemize}
\begin{lstlisting}[style=ShellCmd]
$ git cat-file -p 83baae61804e65cc73a7201a7252750c76066a30 > test.txt
$ cat test.txt
version 1
\end{lstlisting}

\begin{lstlisting}[style=ShellCmd]
$ git cat-file -p 1f7a7a472abf3dd9643fd615f6da379c4acb3e3a > test.txt
$ cat test.txt
version 2
\end{lstlisting}

\begin{itemize}
  \item the objects are just stored on disk
\end{itemize}

\begin{lstlisting}[style=ShellCmd]
$ find .git/objects -type f
.git/objects/d6/70460b4b4aece5915caf5c68d12f560a9fe3e4
.git/objects/83/baae61804e65cc73a7201a7252750c76066a30
.git/objects/1f/7a7a472abf3dd9643fd615f6da379c4acb3e3a
\end{lstlisting}

\begin{itemize}
  \item their type is also stored
\end{itemize}

\begin{lstlisting}[style=ShellCmd]
$ git cat-file -t 1f7a7a472abf3dd9643fd615f6da379c4acb3e3a
blob
\end{lstlisting}

\end{frame}

\begin{frame}[fragile]{Object Storage (4): Trees}
  \begin{itemize}
    \item each node represents a single directory
    \item must contain 1 or more entries (so no empty folders)
    \item includes file names + mode
  \end{itemize}
\begin{lstlisting}[style=ShellCmd]
$ git cat-file -p master^{tree}
100644 blob a906cb2a4a904a152e80877d4088654daad0c859      README
100644 blob 8f94139338f9404f26296befa88755fc2598c289      Rakefile
040000 tree 99f1a6d12cb4b6f19c8655fca46c3ecf317074e0      lib
\end{lstlisting}
\begin{lstlisting}[style=ShellCmd]
$ git cat-file -p 99f1a6d12cb4b6f19c8655fca46c3ecf317074e0
100644 blob 47c6340d6459e05787f644c2447d2595f5d3a54b      simplegit.rb
\end{lstlisting}
\includegraphics[width=0.60\textwidth]{imgs/object_tree}
\end{frame}

\begin{frame}[fragile]{Object Storage (5): The Index as a Tree}
  \begin{itemize}
    \item we can use a tree to represent the index
    \item we can update the tree with the file we created above
  \end{itemize}
\begin{lstlisting}[style=ShellCmd]
$ git update-index --add --cacheinfo 100644 \
  83baae61804e65cc73a7201a7252750c76066a30 test.txt
$ git write-tree
d8329fc1cc938780ffdd9f94e0d364e0ea74f579
$ git cat-file -p d8329fc1cc938780ffdd9f94e0d364e0ea74f579
100644 blob 83baae61804e65cc73a7201a7252750c76066a30      test.txt
\end{lstlisting}

  \begin{itemize}
    \item we can add yet another file to it
  \end{itemize}

\begin{lstlisting}[style=ShellCmd]
$ echo 'new file' > new.txt
$ git update-index test.txt
$ git update-index --add new.txt
$ git write-tree
0155eb4229851634a0f03eb265b69f5a2d56f341
\end{lstlisting}
\end{frame}

\begin{frame}[fragile]{Object Storage (6): The Index as a Tree continued}

  \begin{itemize}
    \item we can also add the same tree as a sub-directory
  \end{itemize}

\begin{lstlisting}[style=ShellCmd]
$ git read-tree --prefix=bak d8329fc1cc938780ffdd9f94e0d364e0ea74f579
$ git write-tree
3c4e9cd789d88d8d89c1073707c3585e41b0e614
$ git cat-file -p 3c4e9cd789d88d8d89c1073707c3585e41b0e614
040000 tree d8329fc1cc938780ffdd9f94e0d364e0ea74f579      bak
100644 blob fa49b077972391ad58037050f2a75f74e3671e92      new.txt
100644 blob 1f7a7a472abf3dd9643fd615f6da379c4acb3e3a      test.txt
\end{lstlisting}

\includegraphics[width=0.60\textwidth]{imgs/index_tree}
\end{frame}

\begin{frame}[fragile]{Object Storage (7): Commit objects}

  \begin{itemize}
    \item we also want to store different commits
    \item each commit contains
    \begin{itemize}
      \item a tree representing the current state
      \item the parent commit
      \item meta-information, such as author and time
      \item (you may get different SHAs because of this)
    \end{itemize}
  \end{itemize}

  \begin{itemize}
    \item we can make a single commit
  \end{itemize}
  
\begin{lstlisting}[style=ShellCmd]
$ echo 'first commit' | git commit-tree d8329f
35f8b9255a9c68f80d90201ae14c39d9c9b66b2a
$ git cat-file -p 35f8b9
tree d8329fc1cc938780ffdd9f94e0d364e0ea74f579
author Tom Wiesing <tkw01536@gmail.com> 1458923656 +0100
committer Tom Wiesing <tkw01536@gmail.com> 1458923656 +0100

first commit
\end{lstlisting}

\begin{itemize}
  \item we can also make commits referencing earlier ones
\end{itemize}

\begin{lstlisting}[style=ShellCmd]
$ echo 'second commit' | git commit-tree 0155eb -p 35f8b9
0d9d54e2c438e22d6656fa1bdca7d76a36d3589c
$ echo 'third commit'  | git commit-tree 3c4e9c -p 0d9d54
970ac2c0207ad51caccce0a71d21283ff7109254
\end{lstlisting}
\end{frame}

\begin{frame}[fragile]{Object Storage (8): Commit objects continued}
  \begin{itemize}
    \item we can now look at the history
  \end{itemize}
\begin{lstlisting}[style=ShellCmd]
$ git log --stat 970ac2
commit 970ac2c0207ad51caccce0a71d21283ff7109254
Author: Tom Wiesing <tkw01536@gmail.com>
Date:   Fri Mar 25 17:38:21 2016 +0100

    third commit

 bak/test.txt | 1 +
 1 file changed, 1 insertion(+)

commit 0d9d54e2c438e22d6656fa1bdca7d76a36d3589c
Author: Tom Wiesing <tkw01536@gmail.com>
Date:   Fri Mar 25 17:38:07 2016 +0100

    second commit

 new.txt  | 1 +
 test.txt | 2 +-
 2 files changed, 2 insertions(+), 1 deletion(-)

commit 35f8b9255a9c68f80d90201ae14c39d9c9b66b2a
Author: Tom Wiesing <tkw01536@gmail.com>
Date:   Fri Mar 25 17:34:16 2016 +0100

    first commit

 test.txt | 1 +
 1 file changed, 1 insertion(+)
\end{lstlisting}
\end{frame}

\begin{frame}[fragile]{Object Storage (9): Commit objects continued}
  \includegraphics[width=0.90\textwidth]{imgs/commit_tree}
\end{frame}
    \begin{frame}[fragile]{References (1): What are REFS?}
  \begin{itemize}
    \item a reference is a pointer to a commit
    \item git stores these as simple files
  \end{itemize}

\begin{lstlisting}[style=ShellCmd]
$ find .git/refs
.git/refs
.git/refs/heads
.git/refs/tags
$ find .git/refs -type f
\end{lstlisting}
  \begin{itemize}
    \item we can write them manually
  \end{itemize}
\begin{lstlisting}[style=ShellCmd]
$ echo '970ac2c0207ad51caccce0a71d21283ff7109254' \
  > .git/refs/heads/master
$ git log --pretty=oneline master
970ac2c0207ad51caccce0a71d21283ff7109254 third commit
0d9d54e2c438e22d6656fa1bdca7d76a36d3589c second commit
35f8b9255a9c68f80d90201ae14c39d9c9b66b2a first commit
\end{lstlisting}
\end{frame}

\begin{frame}[fragile]{References (2): Branches}
  \begin{itemize}
    \item we can use git update-ref instead
  \end{itemize}

\begin{lstlisting}[style=ShellCmd]
$ git update-ref refs/heads/master \
  970ac2c0207ad51caccce0a71d21283ff7109254
\end{lstlisting}

  \begin{itemize}
    \item each branch is just a simple REF (i. e. a pointer)
    \item we can create one easily
  \end{itemize}
\begin{lstlisting}[style=ShellCmd]
$ git update-ref refs/heads/test \
  0d9d54e2c438e22d6656fa1bdca7d76a36d3589c
$ git log --pretty=oneline test
0d9d54e2c438e22d6656fa1bdca7d76a36d3589c second commit
35f8b9255a9c68f80d90201ae14c39d9c9b66b2a first commit
\end{lstlisting}
\end{frame}

\begin{frame}[fragile]{References (3): Branches continued}
  \includegraphics[width=0.90\textwidth]{imgs/branch_tree}
\end{frame}

\begin{frame}[fragile]{References (4): The HEAD}
  \begin{itemize}
    \item the HEAD is a \textit{symbolic} reference to the current branch
    \item each branch has its own HEAD (as you have already seen)
  \end{itemize}
\begin{lstlisting}[style=ShellCmd]
$ cat .git/HEAD
ref: refs/heads/master
$ git checkout test
$ cat .git/HEAD
ref: refs/heads/test
\end{lstlisting}
\begin{itemize}
  \item we could write the symbolic-ref manually
  \item but we can better use
\end{itemize}
\begin{lstlisting}[style=ShellCmd]
$ git symbolic-ref HEAD refs/heads/test
$ cat .git/HEAD
ref: refs/heads/test
\end{lstlisting}
\end{frame}

\begin{frame}[fragile]{References (5): TAGs}
  \begin{itemize}
    \item TAGs are similar to commits
    \item they point to a given version (identified by a commit)
    \item can be light-weight or annotated
    \item we do not go into details here
  \end{itemize}
\end{frame}

\begin{frame}[fragile]{References (6): Remotes}
  \begin{itemize}
    \item remotes are similar to branches
    \item they are considered \textit{read-only}
    \item whenever we fetch / push, they are updated
    \item we will go into remotes in a later section
  \end{itemize}
\end{frame}
    \begin{frame}[fragile]{Merging (1): Branching}
  \begin{itemize}
    \item we have already learned what branches are
    \item we saw how to create them manually
    \item they are intended to be created with
    
  \end{itemize}

\begin{lstlisting}[style=ShellCmd]
$ git branch testing
\end{lstlisting}
  \begin{itemize}
    \item the HEAD of the branch is just pointed to the last commit
    \item ``master'' branch = default name when running ``git init''
  \end{itemize}
  
  \includegraphics[width=0.50\textwidth]{imgs/head_master}
\end{frame}
\begin{frame}[fragile]{Merging (2): Branching (continued)}
  \begin{itemize}
    \item we can also switch the HEAD and update the working directory with
  \end{itemize}

\begin{lstlisting}[style=ShellCmd]
$ git checkout testing
\end{lstlisting}
  \includegraphics[width=0.50\textwidth]{imgs/head_testing}
\end{frame}

\begin{frame}[fragile]{Merging (3): Merging}
  \begin{itemize}
    \item How to bring branches back together?
    \begin{itemize}
      \item Merging or Rebasing
    \end{itemize}
    \item Merging
    \begin{itemize}
      \item via special ``merge commits'' with more than one parent
      \item they merge changes together using different strategies
    \end{itemize}
    \item Rebase
    \begin{itemize}
      \item rewrite (change history) of all commits since the branching point
      \item ensure both versions are preserved
    \end{itemize}
  \end{itemize}
\end{frame}

\begin{frame}[fragile]{Merging (4): How to merge}
  \begin{itemize}
    
    \item fast-forward-merge = merging two HEADs at two different times
    \begin{itemize}
      \item no conflicts, one just has more commits than the other
      \item simply move the merged HEAD to the newest commit
    \end{itemize}
    
    \item ``Recursive'' merge strategy
    \begin{itemize}
      \item default strategy for non-fast-forward merges
      \item if the same line was modified in conflicting ways there will be merge conflicts
    \end{itemize}
    
    \item Other strategies (we do not go into details)
    \begin{itemize}
      \item ours (take ``our'' version for everything)
      \item theirs (take ``their'' version for everything)
      \item octopus (merge more than 2 HEADs)
      \item subtree
      \item \dots
    \end{itemize}
    
    \item Merge conflicts can occur
    \item Git merges would fill a talk on their own
  \end{itemize}
\end{frame}

\begin{frame}[fragile]{Merging (5): Remotes}
  \begin{itemize}
    
    \item remotes = remote branches
    \begin{itemize}
      \item we can retrieve commits from them
      \item we can push commits to them
    \end{itemize}
    
    \item each branch can have an ``upstream'' branch
    \begin{itemize}
      \item they are called remote-tracking branches
      \item they track the state of the remote
      \item of the form remote/branch
      \item ``origin'' is just the default name for the remote
    \end{itemize}
    
    \item you can fetch commits from a remote branch
    \begin{itemize}
      \item ``git fetch REMOTE BRANCH''
      \item this will update the remote reference , create ``FETCH\_HEAD'' and pull all the commit data
      \item afterwards it can be merged into the current branch
      \item ``git merge FETCH\_HEAD''
    \end{itemize}
  \end{itemize}
\end{frame}

\begin{frame}[fragile]{Merging (6): Remotes continued}
  \begin{itemize}
    
    \item ``git pull'' does this in one go
    \begin{itemize}
      \item does some more magic (for example looking up tracked remote)
      \item make sure everything is checked out into the working directory
      \item sometimes it can be simpler to use ``git fetch'' and then ``git merge''
    \end{itemize}
    
    \item we also want to push content to the remote
    \begin{itemize}
      \item update the server with the changes we made
      \item we can use ``git push REMOTE BRANCH''
      \item each branch can be pushed individually or kept local only
      \item only accepts fast-forward pushes
      \item can be forced with a ``--force'' argument
    \end{itemize}
    
    \item other remote operations
    \begin{itemize}
      \item setting the tracking branch (``git branch -u REMOTE/BRANCH'')
      \item deleting a branch from the remote (``git push REMOTE --delete BRANCH'')
    \end{itemize}
  \end{itemize}
\end{frame}
    
    \begin{frame}{Conclusion}
      \begin{itemize}
        \item git is very powerful and useful
        \item git started as a toolkit for a VCS and by now has a more or less user friendly interface
        \item in order to use it properly it is important to understand the underlying model
        \item using git should not be only running memorized commands
      \end{itemize}
    \end{frame}
    
    
    \begin{frame}{The end}
      {\huge
        Thank you for your attention!\\
        Any Questions, Comments, etc?
      }
      
      \begin{itemize}
        \item This talk has been adapted from the Pro Git book (by Scott Chacon and Ben Straub). It is licensed under Creative Commons Attribution Non Commercial Share Alike 3.0 license. 
      \end{itemize}
      % https://git-scm.com/book/en/v2/
    \end{frame}
\end{document}
